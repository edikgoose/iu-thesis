\chapter{Introduction}
\label{ch:intro}
\chaptermark{Optional running chapter heading}

This chapter provides background information on the topic of Site Reliability Engineering (SRE), including its usability and configuration. The chapter discusses load testing, its goals for reliability testing, and its use. In addition, it provides a preview of the content of each subsequent chapter.


\section{Consequences of high load on web services}\label{sec:consequences-of-high-load-on-web-services}
During the software development, it is important to consider the fact that all web services have a load limit. This can be the maximum number of users, or the maximum number of requests per seconds, or something else. After this limit is reached, the service may no longer provide the expected level of quality, which could have a negative impact on the business. For instance, it has been calculated that a one-second latency in the response time of Amazon's services could cost the company \$1.6 billion in sales each year~\cite{one_second_article}.
Most web services become unavailable after reaching their load limit, responding with an error. As a result, they cease to provide the service altogether. It is inevitable, but service can suspend the occurrence of such scenario by applying site-reliability engineering practices.



\section{Site reliability engineering}\label{sec:site-reliability-engineering}
In 2003, Google proposed SRE~\cite{google_sre} practices with the aim to enhance the fault tolerance of web services. These practices include a number of measures, such as defining and measuring reliability targets, monitoring, alerting, and, last but not least, performance optimization. The practices suggest certain reliability patterns that can help to reduce the negative effects of high loads on users.
C. Richardson~\cite{microservices} provides definitions for some of them:
\begin{itemize}
    \item \textbf{Circuit breaker}. Proxy that blocks requests for a specified period of time after the number of consecutive failed attempts exceeds a predetermined threshold.
    \item \textbf{Retry strategy}. If the service returns with an error, the request will be retried after a specified timeout.
    \item \textbf{Rate-limit}. Limit the RPS for a specific client or for all clients.
\end{itemize}

All these reliability patterns should be properly configured. If a software system has many such patterns it can be difficult to maintain them, as the configuration of each one may result in different performance of the system.

\section{Load Testing}\label{sec:load-testing}
To properly configure a reliability patters, it is necessary to test each configuration under high load conditions and all tests should repeat realistic usage scenarios. Load testing~\cite{load_testing_tips} is needed for creation of a manageable high load on a service in order to observe its behavior and identify the maximum point of load. There are three stages of using load testing:

\begin{itemize}
    \item Test Design. Load test scenarios should be as similar to real-world user requests as possible. In most cases, user activity is aggregated into a common metric, typically the number of requests per second.
    \item Test Execution. Load testing requires a significant amount of resources, and therefore, the allocated hardware must be properly configured for high loads. Additionally, load testing tool must be deployed and configured on machine.
    \item Test Analysis. During load testing, a variety of metrics and application logs are collected and aggregated to allow for analysis of the state of the service under stress.
\end{itemize}

Before starting the testing it is necessary to choose the tool for load testing. They are divided into:
\begin{itemize}
    \item Self-hosted. It is deployed on a machine or server by the user themselves. All stages of load testing are controlled by the user. Usually these tools are open source. For example, JMeter~\cite{jmeter} is a self-hosted tool that use your machine resources.
    \item Cloud-based. Such tools provides load testing as a service and hide test execution stage as a black box. Usually cloud-based solutions are paid because they provide load testing as a service, taking over computing power and deploying testing tools. For instance, BlazeMeter~\cite{blazemeter} is a cloud-based solution that needs only the configuration of the load test to execute it.
\end{itemize}

To use cloud-based solutions, it is only necessary to pay money for their use. This is a simple and convenient option, but if the cost is unacceptable, it may be necessary to utilize open source, self-hosted tools. In this scenario, the only expences would be for machine resources. However, deploying and using self-hosted tools correctly requires expertise in DevOps~\cite{devops}, which requires additional funds for specialist resources.


\section{A system for reproducible load testing scenarios}\label{sec:purpose}
It has been previously observed that there are already solutions available for executing load tests. However, if an engineer wants to test their system under the load with different configuration setups, they must deploy and use these tools separately and store in different place the link between the specific system configurations and their load configurations. If, for example, they forget to change system configuration for suitable load test, the incorrect results will be obtained. To avoid this situation it is necessary to develop a system that:
\begin{itemize}
    \item Stores load test and system configurations together as a single scenario.
    \item Reproduces the scenario by setting the system configuration and running the load test.
    \item Stores the results of the executed scenarios.
    \item Can be easily deployed and integrated into existing systems.
\end{itemize}


\section{Structure outline}\label{sec:structure-outline}
Chapter 2 provides an overview of the most popular reliability patterns and their implementation, presents a comparison of different load generation tools. Chapter 3 outlines the architectural design of the system. Chapter 4 presents the implementation of the system and an example of its usage. Chapter 5 discusses difficulties and limitations encountered during development. Chapter 6 concludes the paper with final comments and future work.