\chapter{Literature Review}
\label{ch:lr}
\chaptermark{Second Chapter Heading}

This chapter investigates the reliability patterns and load testing.
It is organized in such way:
\begin{itemize}
    \item~\ref{sec:reliability} gives a brief overview of reliability patterns and their configurations.
    \item~\ref{sec:load_generation} provides the ideas about load testing and the comparison of modern load generation tools.
    \item~\ref{sec:review_conclusion} provides a conclusion of this chapter.
\end{itemize}


\section{Reliability patterns}\label{sec:reliability}
Reliability patterns ensure the stability and resiliency of high-performance systems.
Beyer B. et al.~\cite{google_sre}, examined a number of such strategies, including rate limiting, retry strategy, circuit breaker, and load balancing. Similar concepts are presented in ~\cite{reliability_patterns}, regarding circuit-breaker and retry patterns.
N. C. Mendonca, C. M. Aderaldo, J. Camara and D. Garlan~\cite{circuit_breaker} outlines that circuit-breaker can have several tuning parameters, such as the sleep window, the error percentage threshold, and others. But in real implementations, the set of tuning params may vary. This set may depend on the specific implementation or features of the programming language.

\begin{longtable}[c]{|p{4.5cm}|p{5.2cm}|p{5.3cm}|}
    \caption{Circuit breaker parameters for different implementations}
    \label{tab:patterns_table} \\
    \hline
    \textbf{Abstract parameter} & \textbf{Java (Resilience4j)} & \textbf{Python (Circuit breaker library)} \\
    \endhead
    \hline
    Timeout                     & maxWaitDuration              & recovery\_timeout                         \\
    \hline
    Error Percent Threshold     & failureRateThreshold         & failure\_threshold                        \\
    \hline
    Request Volume threshold    & minimumNumberOfCalls         & -                                         \\
    \hline
    Sleep window                & slidingWindowSize            & -                                         \\
    \hline
    - & \begin{tabular}[c]{@{}l@{}}
            recordExceptions,\\ slidingWindowType
    \end{tabular} & \begin{tabular}[c]{@{}l@{}}
                        expected\_exception,\\ fallback\_function
    \end{tabular} \\
    \hline
\end{longtable}

Table~\ref{tab:patterns_table} shows accordance of configuration parameters for abstract circuit breaker~\cite{circuit_breaker} and for real implementation in Java~\cite{resilience4j} and Python~\cite{circuitbreaker}. It shows that in real implementation we can have incomplete or more enhanced set of parameters and it's hard to create generic number of configuring params for all implementations.


\section{Load testing}\label{sec:load_generation}
To properly select a load generation tool, it is necessary to understand the goals of load testing. Daniel A.Menasc`e~\cite{load_testing_base} explains that load testing allows to measure the performance of a system: the average response time and throughput as a certain number of concurrent requests. Therefore, it is crucial for a load testing tool to generate a specified number of requests per second (RPS) and presents the results as the response time and the maximum RPS that the server can handle.
R. Abbas, Z. Sultan, and S. N. Bhatti~\cite{load_testing_tools} provide a comprehensive comparison of modern load generation tools, using various criteria, including different protocol support, costs, and ease of use. They compared such tools as JMeter~\cite{jmeter}, Siege~\cite{siege}, etc. Additionally, S. Myasnikov and D. Namiot~\cite{load_testing_tools_rus} present a similar comparison, but with an extended number of tools, such as Yandex Tank~\cite{yandex_tank} and Gatling~\cite{gatling}. All these tools meet the conditions specified by Daniel A.Menasc`e, and they differ from each other in terms of the implementation and the number of additional features. Table~\ref{tab:load_tools} shows a compiled comparison of open-source tools of the discussed studies.
All tools are suitable for this specific scenario. For example, if it is necessary to quickly execute a simple load test, Siege~\cite{siege} can be used. At the same time, Jmeter~\cite{jmeter} can be used for complex and several parallel tests. However, these tools treat the system under the test as a black box, and do not have any knowledge of its implementation or configuration. Therefore, they are unable to show or change the configuration of reliability patterns during the testing.

\begin{longtable}[c]{|p{4cm}|p{2.5cm}|p{2.5cm}|p{2.5cm}|p{2.5cm}|}
    \caption{Comparison of load testing tool}
    \label{tab:load_tools} \\
    \hline
    \textbf{Feature}        & \textbf{JMeter}             & \textbf{Yandex Tank} & \textbf{Gatling}                          & \textbf{Siege}                      \\
    \endhead
    \hline
    Application support     & Databases, Web              & Web                  & Databases, Web                            & Web                                 \\
    \hline
    Built-in analyzing tool & Yes                         & No                   & No                                        & Yes                                 \\
    \hline
    Interface               & GUI                         & Config files         & Config files, GUI                         & CLI                                 \\
    \hline
    Drawbacks               & Consumes much time to setup & Supports only HTTP   & Require knowledge of programming language               & Supports only the simplest scenario              \\
    \hline
    Platform support        & Everything with JVM         & UNIX                 & Everything with JVM                       & UNIX                                \\
    \hline
\end{longtable}


\section{Conclusion}\label{sec:review_conclusion}
The implementation of reliability patterns varies from library to library, so, it is important to consider all the tune parameters of a specific implementation and use abstract ideas only to understand the work of the pattern.
Most of the discussed load testing tools can be used to find performance metrics for the system, but these tools cannot be integrated into the tested system and cannot read or change its state.